\usepackage{hyperref}
\usepackage{eso-pic}
\usepackage[german=swiss]{csquotes}
\usepackage{pdfpages}
\usepackage[skip = 4pt, font=large, labelfont={large}, justification=centering, singlelinecheck=false]{caption}
\usepackage{float}
\usepackage{makecell}
\usepackage{multicol}
\usepackage{threeparttable}
\usepackage{tabularx,booktabs}

%
% %%%%%%%%%%%%%%%%%%%%%  Größen ändern gegen LaTeXs Hysterie %%%%%%%%%%%%%%%%%%%%%%%%%%%%%%
\widowpenalty=10000 % erschwert Hurenkindern das Leben
\clubpenalty=3000 % erschwert Schusterjungen das Leben
\tolerance=500  %Wortabstände dehnbarer; fürs Final auf 300!!
\hbadness= 500
  \emergencystretch= 1.5em
 \hfuzz = 0.3pt
 \vfuzz \hfuzz
 \raggedbottom
%
%% erste Zeile einrücken
\setlength{\parindent}{1em}

\usepackage{multicol}

\setlength{\abovecaptionskip}{10pt plus 3pt minus 2pt}
\setlength{\belowcaptionskip}{12pt plus 3pt minus 2pt}


%
% %% Fußnoten
\usepackage{footnote}
\usepackage[stable, norule, flushmargin]{footmisc} %% Fußgut
%\usepackage{float}

\counterwithout{footnote}{chapter} % über Kapitel hinweg zählen
\addtolength{\skip\footins}{1ex plus 2mm} % Abstand zum Text
\renewcommand{\footnotesep}{1ex} % da nummeriert, Abstand ok
% rechtsbündig aus likem Rand

\let\Umathcode\XeTeXmathcode \let\Umathchardef\XeTeXmathchardef
%
% %____________ Größen Ende ______________________________
%
%
% \newcommand{\on}[1]{{#1}} %deaktiviertes oldstylenums
% \newcommand{\p}{{\npnoaddmissingzero \npdecimalsign{.}}} %p-Wert
%                                 %kompakt in zB Kovarianztabellen
%
\AtBeginDocument{%
  \expandafter\def\expandafter\@tabarray\expandafter{%
    \expandafter\CT@start\@tabarray}}
%
% %%%%%%%%%      Umgebungen und Kommandos   %%%%%%%%%%%%%%%%%%%%%%%%%%%%
%
% \newcommand{\hoch}[1]{\textsuperscript{#1}} % \hoch{2}1986  wird ²1986
%
% %% Literaturverzeichnis
% \newenvironment{literatur}
% {\begin{description}\footnotesize %\raggedright
%   \setlength{\itemsep}{0ex plus0.0ex} \setlength{\parsep}{0ex plus 0ex
%   minus 0ex}
% }
%   {\end{description}}%% Literatur Ende
%
%%%%% Farbboxen für Hinweise ------
\usepackage{tcolorbox}
\tcbset{textmarker/.style={%
        enhanced,
        parbox=false,boxrule=0mm,boxsep=0mm,arc=0mm,
        outer arc=0mm,left=6mm,right=3mm,top=7pt,bottom=7pt,
        toptitle=1mm,bottomtitle=1mm,oversize}}

\newtcolorbox{hintBox}{textmarker,
    borderline west={6pt}{0pt}{yellow},
    colback=yellow!10!white}

\newtcolorbox{importantBox}{textmarker,
    borderline west={6pt}{0pt}{red},
    colback=red!10!white}

\newtcolorbox{noteBox}{textmarker,
    borderline west={6pt}{0pt}{green},
    colback=green!10!white}

\newcommand{\Anmerkung}[1]
{\begin{hintBox}
#1
\end{hintBox}}

\newcommand{\Alarm}[1]
{\begin{importantBox}
#1
\end{importantBox}}

\newcommand{\Allesgut}[1]
{\begin{noteBox}
#1
\end{noteBox}}

\IfFileExists{microtype.sty}{% use microtype if available
  \usepackage[]{microtype}
  \UseMicrotypeSet[protrusion]{basicmath} % disable protrusion for tt fonts
}{}

\usepackage{mdframed}

\mdfdefinestyle{MyFrame}{%
    linecolor=black,
    outerlinewidth=2pt,
    %roundcorner=20pt,
    innertopmargin=4pt,
    innerbottommargin=4pt,
    innerrightmargin=4pt,
    innerleftmargin=4pt,
        leftmargin = 4pt,
        rightmargin = 4pt
    %backgroundcolor=gray!50!white}
        }

% %% Zitate \zitat{Autor 1986:5}{Text} Autor steht rechtsbündig in der
% %% letzten Zeile oder, wenns nicht passt rechtsbündig drunter
% \newenvironment{zitat}[2]{%
% \begin{list}{}{%
% \setlength{\leftmargin}{2em}%
% \setlength{\rightmargin}{2em}%
% \setlength{\itemindent}{0cm}%
% \setlength{\labelwidth}{0cm}%
% \setlength{\parsep}{0cm}\setlength{\listparindent}{2mm}}%
% \small
% \begin{spacing}{1}%
% \item  #2%
% %\scshape
% \unskip %
% \hfill %
% \quad%
% \nolinebreak[3]{%
% \hspace*{\fill}%
% \nolinebreak%
% \mbox{(#1)}}%
% \end{spacing}%
% \end{list}%
% \vspace{-\topsep}{}}
%
%% \PBS für geschützten Backslash
\newcommand{\PreserveBackslash}[1]{\let\temp=\\#1\let\\=\temp}
\let\PBS=\PreserveBackslash
%
\renewcommand{\labelitemi}{--}
%
% %%% Bilder einbauen (breitbild = Seitenbreite skalieren; bild zentriert
% \newcommand{\breitbild}[2]{\includegraphics[width=\linewidth]{#1} \captionbelow{#2}}
% \newcommand{\hist}[2]{\centering\includegraphics[width=.6\textwidth]{#1} \captionbelow{#2}}
% \newcommand{\bild}[2]{\centering \includegraphics{#1} \captionbelow{#2}}
% \newcommand{\seite}[1]{\centering\vspace{-1cm}\hspace{-1.7cm} \includegraphics[width=1.22\linewidth]{#1}}
% \newcommandtwoopt{\Bild}[3][1][0 0 0
% 0]{ \begin{center}
%     \includegraphics[width=#1\linewidth,trim = #2, clip]{Bilder/#3}
%   \end{center}}
% %%%% Aufzählungen
%
\renewcommand{\labelitemi}{\small\textbullet}
%
% \newcommand{\Beschrieb}[1]{\ifthenelse{\isundefined{#1}}{}{
% \begin{minipage}[H]{1.0\linewidth}
%   \begin{multicols}{3}
%  #1
%  \end{multicols}
% \end{minipage}\vspace{5mm}}}
%
% %__________________ Umgebungen Ende _____________________________________________
%
% %--------  Tabellenbefehle  ----------------------
% %LaTeX interpretiert Tabulatoren und Zeilenenden für Tabellen
% % & und \\ können eingesetzt werden, wenn nötig
% % bei zB \\\midrule%! muss das Zeilenende auskommentiert sein
%
\renewcommand{\floatpagefraction}{.7}
\renewcommand{\textfraction}{.12}
\renewcommand{\topfraction}{.8}     % vorher: .7
\renewcommand{\bottomfraction}{.5}  % vorher: .3
\makeatletter
\renewcommand{\fps@figure}{htbp}
\renewcommand{\fps@table}{htbp}
\makeatother
%
% \setbox0=\hbox{%
%   \begin{tabular}{c}
%  \global\let\CsvNewline\\%
%  \end{tabular}}
%  {\catcode`\^^M=\active%
%   \gdef\CsvObeylines{\catcode`\^^M=\active \let^^M=\CsvNewline}}%
%
% \newlength{\cslhangindent}
% \setlength{\cslhangindent}{1.5em}
% \newenvironment{CSLReferences}%
% {\setlength{\parindent}{0pt}%
% \everypar{\setlength{\hangindent}{\cslhangindent}}\ignorespaces}%
% {\par}
%
\setcapindent{1em}
%
\usepackage[ngerman]{varioref}

\newcommand{\TBZeilenabstand}{\aboverulesep = 0pt \belowrulesep = 0mm}

\newcommand{\NormalZeilenabstand}{\aboverulesep = 0.605mm \belowrulesep = 0.984mm}


% %% variable Verweise
\renewcommand\reftextfaceafter{gegen\"uberliegend}
\renewcommand\reftextfacebefore{gegen\"uberliegend}
\renewcommand\reftextbefore{vorherige Seite}
\renewcommand\reftextafter{n\"achste Seite}
\renewcommand\reftextcurrent{\unskip}
\renewcommand\reftextfaraway[1]{Seite~\pageref{#1}}
%
\makeatletter
\@addtoreset{figure}{section}
\@addtoreset{table}{section}
\makeatother
%
% %\definecolor{darkestblue}{rgb}{1,27,84}

\usepackage[ngerman]{cleveref}

\renewcommand{\thefigure}{\thesection.\arabic{figure}}

\renewcommand{\thetable}{\thesection.\arabic{table}}

% \urlstyle{same} % nur für APA-Style

% KOMA options

\KOMAoptions{pagesize, paper=portrait}
      \newgeometry{a4paper, lmargin=30mm, top=30mm, textwidth=15.5cm,
 textheight=22.5cm}%
 
\floatplacement{figure}{H}
\floatplacement{table}{H}

\makeatletter
\newcommand{\blandscape}{
\begin{landscape}
%\KOMAoptions{pagesize, paper=landscape}
    }

\newcommand{\elandscape}{%
\end{landscape}
%\KOMAoptions{pagesize, paper=portrait}
%      \newgeometry{a4paper, lmargin=30mm, top=30mm, textwidth=15.5cm,
% textheight=22.5cm}%
    }
\makeatother

\newcommand{\btwocol}{\begin{multicols}{2}}
\newcommand{\etwocol}{\end{multicols}}

\newcommand{\pdf}[3][1cm 1cm 1cm 1cm]
  {\begin{center} \includegraphics[width=\linewidth,trim = #1,
      clip, page = #3]{#2}
   \end{center}}

